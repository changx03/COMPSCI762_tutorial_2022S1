\documentclass[11pt]{article}

\usepackage{hyperref}
\usepackage[inline]{enumitem}
\usepackage{booktabs}
\usepackage{multirow}
\usepackage[flushleft]{threeparttable}

\makeatletter
\def\@seccntformat#1{%
  \expandafter\ifx\csname c@#1\endcsname\c@section\else
  \csname the#1\endcsname\quad
  \fi}
\makeatother

\topmargin -.5in
\textheight 9in
\oddsidemargin -.25in
\evensidemargin -.25in
\textwidth 7in

\sloppy

\begin{document}

\title{COMPSCI 762 2022 S1 Week 8 Questions -- Artificial Neural Networks}
\author{Luke Chang}

\maketitle

\medskip

\section{Question 1}
\label{q1}

There are four main activation functions we have discussed in the class:
step function, linear function, sigmoid function, and rectified linear function.
For each of them explain what their pros and cons are and what kind of models they can learn.

\section{Question 2}
\label{q2}

In neural networks, we talked about getting stuck in a local minima. If you got stuck
in a local minima would that be overfitting or underfitting or could it be either?

\section{Question 3}
\label{q3}

If you have a fully connected neural network with 3 features, 1 hidden layer with 2 nodes,
and an output layer with 5 nodes.
Recall a weight is associated with an edge between two nodes.

\begin{enumerate}
  \item How many weights will you learn?
  \item What will be the form of the hypothesis returned by this neural network algorithm?
  \item If you add a second hidden layer with 2 nodes, how many more numbers will there be in your hypothesis?
  \item What is the size of the set of all possible hypotheses?
  \item What activation function will you use on each layer? What loss function will you use? Why do you choose them?
\end{enumerate}

\vspace{18pt}
\noindent
\textbf{Note:}\\
The answer for each question should be 100 words or less. (Question 3 has 5 questions.) You may include figures.

\end{document}
