\documentclass[11pt]{article}

\usepackage{hyperref}
\usepackage[inline]{enumitem}
\usepackage{booktabs}
\usepackage{multirow}
\usepackage[flushleft]{threeparttable}

\makeatletter
\def\@seccntformat#1{%
  \expandafter\ifx\csname c@#1\endcsname\c@section\else
  \csname the#1\endcsname\quad
  \fi}
\makeatother

\topmargin -.5in
\textheight 9in
\oddsidemargin -.25in
\evensidemargin -.25in
\textwidth 7in

\sloppy

\begin{document}

\title{COMPSCI 762 2022 S1 Week 9 Questions -- Artificial Neural Networks}
\author{Luke Chang}

\maketitle

\medskip

\section{Question 1}
\label{q1}
Designing artificial neural network models to solve boolean functions:

\begin{enumerate}
    \item Design a two-point preceptron (an artificial neuron) that implements the boolean function $A \land \neg B$.
    \item Design a two-layer network of perceptrons that implements $A \oplus B$ (XOR logic).
\end{enumerate}


\section{Question 2}
\label{q2}

Table \ref{tab1} consists of training data from an employee database.
The data have been generalized.
For example, “31 . . . 35” for age represents the age range of 31 to 35.
For given row entry, count represents the number of data tuples having the
values for department, status, age and salary given in that row. 
Let \textit{Status} be the class-label attribute.


\begin{table}[h!]
    \centering
    \begin{tabular}{rlllr|c}
        \toprule
        Index & Department & Age         & Salary        & Count & Status \\
        \midrule
        1     & sales      & 31 . . . 35 & 46K . . . 50K & 30    & senior \\
        2     & sales      & 26 . . . 30 & 26K . . . 30K & 40    & junior \\
        3     & sales      & 31 . . . 35 & 31K . . . 35K & 40    & junior \\
        4     & systems    & 21 . . . 25 & 46K . . . 50K & 20    & junior \\
        5     & systems    & 31 . . . 35 & 66K . . . 70K & 5     & senior \\
        6     & systems    & 26 . . . 30 & 46K . . . 50K & 3     & junior \\
        7     & systems    & 41 . . . 45 & 66K . . . 70K & 3     & senior \\
        8     & marketing  & 36 . . . 40 & 46K . . . 50K & 10    & senior \\
        9     & marketing  & 31 . . . 35 & 41K . . . 45K & 4     & junior \\
        10    & secretary  & 46 . . . 50 & 36K . . . 40K & 4     & senior \\
        11    & secretary  & 26 . . . 30 & 26K . . . 30K & 6     & junior \\
        \bottomrule
    \end{tabular}
    \caption{Employee database}
    \label{tab1}
\end{table}

\begin{enumerate}
    \item Design a multilayer feed-forward neural network for the given data. 
        Label the nodes in the input and output layers.
        \label{q2.1}
    \item  Using the multilayer feed-forward neural network obtained in (\ref{q2.1}), show the weight values after one
    iteration of the back-propagation algorithm, given the training instance “(sales, senior, 31. . . 35,
    46K,. . . 50K)”. Indicate your initial weight values and biases and the learning rate used.
    
\end{enumerate}

\textbf{Note:}\\
All questions should be solved by hand. You may use any plotting tools/packages,
but you should use neither PyTorch nor Tensorflow to build neural network models.
\end{document}
