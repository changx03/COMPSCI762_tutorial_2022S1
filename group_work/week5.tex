\documentclass[11pt]{article}

\usepackage{hyperref}
\usepackage[inline]{enumitem}
\usepackage{booktabs}
\usepackage{multirow}
\usepackage[flushleft]{threeparttable}

\makeatletter
\def\@seccntformat#1{%
  \expandafter\ifx\csname c@#1\endcsname\c@section\else
  \csname the#1\endcsname\quad
  \fi}
\makeatother

\topmargin -.5in
\textheight 9in
\oddsidemargin -.25in
\evensidemargin -.25in
\textwidth 7in

\sloppy

\begin{document}

\title{COMPSCI 762 2022 S1 Week 5 Questions -- Regression \& Preprocessing}
\author{Luke Chang}

\maketitle

\medskip

\section{Question 1 -- Regression}
Consider a time series data as the following:

\begin{table}[h]
  \centering
  \begin{tabular}{c|llllllll}
  X & 8 & 3  & 2 & 10 & 11 & 3 & 6 & 5  \\
  Y & 4 & 12 & 1 & 12 & 9  & 4 & 9 & 6 
  \end{tabular}
\end{table}

\begin{enumerate}
  \item Use the least square method to determine the equation of line of best fit for the data. Then plot the line.
  \item Given a new data point $X=9$, what is the predicted value of Y?
\end{enumerate}


\section{Question 2 -- Data Cleaning}

You sent a survey to collect data about customers who buy lunch in the university cafe.
Given the following data for the attribute -- \textit{Age}:
\[13,15,16,16,19,20,20,21,22,22,25,25,25,25,30,33,33,35,35,35,35,36,40,45,46,52,70\]

\begin{enumerate}
  \item Use \textit{smoothing by bin means} to smooth these data with 3 bins. 
  Illustrate your steps. Comment on the effect of this technique for the given data.
  \item How do you determine \textit{outliers} in the data?
  \item What other methods are there for data smoothing?
\end{enumerate}

\section{Question 3 -- Preprocessing on Real-World Data}
Your team decides to participate a Kaggle competition on predicting car price.
Here is the URL for the Kaggle page: \url{https://www.kaggle.com/ersany/car-price-prediction?resource=download&select=car_price.csv}.

\begin{enumerate}
  \item What preprocessing techniques would you apply in order to train a regression model?
  \item There is no missing value in the dataset. What if $20\%$ of \texttt{body\_type} are missing? What imputation technique should you use?
  \item What if $20\%$ of \texttt{body\_color} are missing? What imputation technique should you use? Can you use the same imputation technique for both attributes?
\end{enumerate}

\textbf{Note:} The documentation of \texttt{scikit-learn} preprocessing is a good starting point.
The link can be found here: \url{https://scikit-learn.org/stable/modules/preprocessing.html#preprocessing}.
Writing code for Question 3 can certainly help you to determine the optimal solution. However, it is not required. You must explain why a
particular technique has been selected.



\end{document}
