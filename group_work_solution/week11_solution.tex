\documentclass[11pt]{article}

\usepackage{hyperref}
\usepackage[inline]{enumitem}
\usepackage{booktabs}
\usepackage{multirow}
\usepackage[flushleft]{threeparttable}
\usepackage{amsmath}
\usepackage{xcolor}

\makeatletter
\def\@seccntformat#1{%
  \expandafter\ifx\csname c@#1\endcsname\c@section\else
  \csname the#1\endcsname\quad
  \fi}
\makeatother

\topmargin -.5in
\textheight 9in
\oddsidemargin -.25in
\evensidemargin -.25in
\textwidth 7in

\sloppy

\begin{document}

\title{COMPSCI 762 2022 S1 Week 11 Questions}
\author{Luke Chang}

\maketitle

\medskip

\section{Question 1}
\label{q1}
Answer the following questions regard to ensemble methods. The explanation for each question should be 200 words or less.
Using figures to illustrate your solution is highly encouraged.

\begin{enumerate}
    \item What are the two key factors an ensemble must have?

    {\color{blue}
        \begin{enumerate}
            \item Each model must perform better than random guess;
            \item Must be uncorrelated;
        \end{enumerate}
    }
    \item Bagging changes two thins in a dataset. What are they?

    {\color{blue}
        \begin{enumerate}
            \item The choice of data instances;
            \item The distribution over them (with replacement);
        \end{enumerate}
    }
    \item What is the main differences between {\em random forest} (RF), bagging and XGBoost?

    {\color{blue}
        XG Boost chooses without replacement so does not change the distribution of the dataset;
        Also Boosting will have to use a smaller training set by definition; 
        RF uses democratic voting and XG Boost uses weighted voting.
    }
    \item Which of the ``methods for constructing ensembles'' do RF and XGBoost use?

    {\color{blue}
        They are both manipulating the training set, manipulating the input features (columns), injecting Randomness.
    }
    \item Will variable importance in RF always give you the the ``correct'' answer? Why?

    {\color{blue}
    No, because if you have correlated attributes Random forest will say neither are important even if they are the most important.\\
    This is because it randomizes the variables one at a time, thereby relying on the correlated variable when each is randomized.\\
    \textbf{Example:}\\
    Given a logic Function: $A \lor (B \land C)$ - B and C are correlated. If we train a tree with only A and B, or A and C, we will not have the correct logic.
    }
    \item Between RF and bagging, what will the effect be of having a data set with a larger or smaller number of instances?

    {\color{blue}
        The effect on bagging and random forest will be the same.
        They both sample the instance space with replacement.
    }
    \item Between RF and bagging, what will the effect be of having a data set with a larger or smaller number of features?

    {\color{blue}
    Since random forests sample the features, you might get better results when there are a lot of
    features because you got rid of a lot of noise, but with a data set with only a few features you
    might do worse because you are not left with enough features to make a good classifier.
    }
\end{enumerate}

\section{Question 2}
\label{q2}

No standard solution for Question 2. Any reasonable analysis is a valid solution.

\end{document}
