\documentclass[10pt]{article}

\usepackage[english]{babel}
\usepackage[linguistics]{forest}
\usepackage[utf8]{inputenc}
\usepackage{algorithmic}
\usepackage{amsfonts}
\usepackage{amsmath}
\usepackage{amssymb}
\usepackage{array}
\usepackage{bookmark}
\usepackage{caption}
\usepackage{colortbl}
\usepackage{csquotes}
\usepackage{graphicx}
\usepackage{hyperref}
\usepackage{lipsum}
\usepackage{lmodern}
\usepackage{mathptmx}
\usepackage{mathtools}
\usepackage{multirow}
\usepackage{pgfplots}
\usepackage{svg}
\usepackage{xcolor}
\usepackage{multicol}
\usepackage{float}

\usetikzlibrary{calc}

\DeclareMathOperator*{\argmax}{argmax}

\makeatletter
\def\@seccntformat#1{%
  \expandafter\ifx\csname c@#1\endcsname\c@section\else
  \csname the#1\endcsname\quad
  \fi}
\makeatother

\topmargin -.5in
\textheight 9in
\oddsidemargin -.25in
\evensidemargin -.25in
\textwidth 7in

\sloppy

\begin{document}

\title{COMPSCI 762 2022 S1 Week 4 Solution}
\author{Luke Chang}

\maketitle

\medskip

\section{Question 1}
\label{q1}

\begin{itemize}
  \item Check Week 2 lecture slides from page 40 to 48.
\end{itemize}

\section{Question 2}
\label{q2}

\begin{itemize}
  \item The weather is either Shower or Clear. This is a binary classification task. Let Shower be \textbf{P}ositive and Clear be \textbf{N}egative:
    \begin{table}[H]
      \centering
      \small
      \begin{tabular}{ccccc}
      \textbf{}                        & \textbf{}                       & \multicolumn{2}{c}{\textbf{Predicted}}          &                \\
                                      &                                 & \textbf{P}             & \textbf{N}             & \textbf{Total} \\ \cline{3-4}
      \multirow{2}{*}{\textbf{Actual}} & \multicolumn{1}{c|}{\textbf{P}} & \multicolumn{1}{c|}{4} & \multicolumn{1}{c|}{1} & 5              \\ \cline{3-4}
                                      & \multicolumn{1}{c|}{\textbf{N}} & \multicolumn{1}{c|}{3} & \multicolumn{1}{c|}{2} & 5              \\ \cline{3-4}
                                      & \textbf{Total}                  & 7                      & 3                      & \textbf{10}   
      \end{tabular}
    \end{table}
  \item $\text{Acc.} = \frac{6}{10} = 0.6$
  \item $\text{Precision (P)} = \frac{\text{TP}}{\text{TP}+ \text{FP}} = \frac{4}{4+3} \approx 0.571$
  \item $\text{Recall (R)} = \frac{\text{TP}}{\text{TP}+ \text{FN}} = \frac{4}{4+1} \approx 0.8$
  \item $F_1 = 2 \frac{P \times R}{P + R} = 2 \times \frac{0.571 \times 0.8}{0.571 + 0.8} \approx 0.667$
  \item \textbf{Note: } A model with high Recall may also has high FPR (Type I Error).
  \item ROC curve:
    \begin{table}[H]
      \centering
      \small
      \begin{tabular}{cc|cccccc}
      \textbf{}      & \textbf{}           & \multicolumn{6}{l}{\textbf{Thresholds}}                                             \\
      \textbf{Class} & \textbf{Prediction} & \textbf{0} & \textbf{0.2} & \textbf{0.4} & \textbf{0.6} & \textbf{0.8} & \textbf{1} \\ \hline
      P              & 0.95                & 1          & 1            & 1            & 1            & 1            & 0          \\
      N              & 0.85                & 1          & 1            & 1            & 1            & 1            & 0          \\
      P              & 0.78                & 1          & 1            & 1            & 1            & 0            & 0          \\
      P              & 0.66                & 1          & 1            & 1            & 1            & 0            & 0          \\
      N              & 0.6                 & 1          & 1            & 1            & 1            & 0            & 0          \\
      P              & 0.55                & 1          & 1            & 1            & 0            & 0            & 0          \\
      N              & 0.53                & 1          & 1            & 1            & 0            & 0            & 0          \\
      N              & 0.52                & 1          & 1            & 1            & 0            & 0            & 0          \\
      N              & 0.51                & 1          & 1            & 1            & 0            & 0            & 0          \\
      P              & 0.4                 & 1          & 1            & 1            & 0            & 0            & 0         
      \end{tabular}
    \end{table}
  \item Counting TP and FP:
    \begin{table}[H]
      \centering
      \small
      \caption{Counting TP and FP}
      \begin{tabular}{c|cccccc}
      \textbf{Threshold} & \textbf{0} & \textbf{0.2} & \textbf{0.4} & \textbf{0.6} & \textbf{0.8} & \textbf{1} \\ \hline
      \textbf{TPR}       & 1          & 1            & 1            & 0.60         & 0.2          & 0          \\
      \textbf{FPR}       & 1          & 1            & 1            & 0.4          & 0.2          & 0         
      \end{tabular}
    \end{table}
  \item Sort the results:
    \begin{table}[H]
      \centering
      \small
      \caption{Sort the results}
      \begin{tabular}{c|cccccc}
      \textbf{Threshold} & \textbf{1} & \textbf{0.8} & \textbf{0.6} & \textbf{0.4} & \textbf{0.2} & \textbf{0} \\ \hline
      \textbf{TPR}       & 0          & 0.2            & 0.6            & 1         & 1         & 1          \\
      \textbf{FPR}       & 0          & 0.2           & 0.4           & 1          & 1          & 1         
      \end{tabular}
    \end{table}
  \item Final ROC plot:
    \begin{figure}[H]
      \centering
      \includegraphics[width=0.4\columnwidth]{roc_curve_eg.pdf}
    \end{figure}
  \item AUC is the area under the ROC curve.
\end{itemize}

\section{Question 3}
\label{q3}

\begin{itemize}
  \item After we find the optimal parameters for the model using CV, we have to train the model one more time with all data points. 
  \item We don't want to do a train-test split. We want to use all the data.
  \item The actual test set is hidden from us. When we submit the model, Kaggle will report a ranking score. However, we can not access the test set. 
\end{itemize}

\end{document}
