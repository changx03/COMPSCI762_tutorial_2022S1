\documentclass[11pt]{article}

\usepackage[flushleft]{threeparttable}
\usepackage[inline]{enumitem}
\usepackage{algorithmic}
\usepackage{amsfonts}
\usepackage{amsmath}
\usepackage{amsmath}
\usepackage{amssymb}
\usepackage{array}
\usepackage{bm}
\usepackage{booktabs}
\usepackage{caption}
\usepackage{colortbl}
\usepackage{csquotes}
\usepackage{graphicx}
\usepackage{heuristica}
\usepackage{hyperref}
\usepackage{mathptmx}
\usepackage{multirow}
\usepackage{pgfplots}
\usepackage{siunitx}
\usepackage{subcaption}
\usepackage{svg}
\usepackage{tabularx}
\usepackage{textcomp}
\usepackage{xcolor}

\makeatletter
\def\@seccntformat#1{%
  \expandafter\ifx\csname c@#1\endcsname\c@section\else
  \csname the#1\endcsname\quad
  \fi}
\makeatother

\topmargin -.5in
\textheight 9in
\oddsidemargin -.25in
\evensidemargin -.25in
\textwidth 7in

\sloppy

\begin{document}

\title{COMPSCI 762 2022 S1 Week 12 Solutions}
\author{Luke Chang}

\maketitle

\medskip

\section{Question 1 -- Clustering}
\label{q1}

\begin{figure}[h]
    \centering
    \small
    \caption{A data set has 8 instances, from A1 to A8. The Euclidean distances between every two instances are given as the following:}
    \includegraphics[width=0.5\textwidth]{../group_work/k-mean_review_q1.png}
    \label{fig1.1}
\end{figure}

Use the data in Figure~\ref{fig1.1} to answer the questions below:

\subsection{}
Suppose that the initial seeds (centers of each cluster) are \textbf{A1}, \textbf{A4} and \textbf{A7}. 
Run the k-means algorithm for 1 epoch only. At the end of this epoch show the new clusters (i.e. the examples belonging to each cluster)

\noindent
\textbf{Answer:}

\textbf{C1} is centred at \textbf{A1}, \textbf{C2} is centred at \textbf{A4}, and \textbf{C3} is centred at \textbf{A7}.\\
For each point:\\
\begin{itemize}
    \item $\textbf{A1} \rightarrow \textbf{C1}$
    \begin{table}[h!]
        \scriptsize
        \begin{tabular}{ccc}
        Item & Centroid & Dist \\ \hline
        A2   & A1       & $\sqrt{45}$ \\
        A2   & A4       & $\sqrt{49}$ \\
        A2   & A7       & $\sqrt{5}$
        \end{tabular}
    \end{table}
    \item $\textbf{A2} \rightarrow \textbf{C3}$
    \begin{table}[h!]
        \scriptsize
        \begin{tabular}{ccc}
        Item & Centroid & Dist \\ \hline
        A3   & A1       & $\sqrt{63}$ \\
        A3   & A4       & $\sqrt{11}$ \\
        A3   & A7       & $\sqrt{47}$
        \end{tabular}
    \end{table}
    \item $\textbf{A3} \rightarrow \textbf{C2}$
    \item $\textbf{A4} \rightarrow \textbf{C2}$
\end{itemize}

\begin{table}[h!]
    \scriptsize
    \begin{tabular}{ccc}
    Item & Centroid & Dist \\ \hline
    A5   & A1       & $\sqrt{41}$ \\
    A5   & A4       & $\sqrt{2}$ \\
    A5   & A7       & $\sqrt{21}$
    \end{tabular}
\end{table}
\begin{itemize}
    \item $\textbf{A5} \rightarrow \textbf{C2}$
    \begin{table}[h!]
        \scriptsize
        \begin{tabular}{ccc}
        Item & Centroid & Dist \\ \hline
        A6   & A1       & $\sqrt{28}$ \\
        A6   & A4       & $\sqrt{7}$ \\
        A6   & A7       & $\sqrt{13}$
        \end{tabular}
    \end{table}
    \item $\textbf{A6} \rightarrow \textbf{C2}$
    \item $\textbf{A7} \rightarrow \textbf{C3}$
    \begin{table}[h!]
        \scriptsize
        \begin{tabular}{ccc}
        Item & Centroid & Dist \\ \hline
        A8   & A1       & $\sqrt{6}$ \\
        A8   & A4       & $\sqrt{5}$ \\
        A8   & A7       & $\sqrt{53}$
        \end{tabular}
    \end{table}
    \item $\textbf{A8} \rightarrow \textbf{C2}$
\end{itemize}

The new clusters: $C1=\{A1\},C2=\{A3, A4, A5, A6, A8\}, C3=\{A2, A7\}$

\subsection{}
Use single-linkage (MIN) agglomerative clustering to group the data. Show the dendrogram.

\noindent
\textbf{Answer:}

\begin{table}[h!]
    \centering
    \scriptsize
    \caption{Step 1}
    \begin{tabular}{c|cccc|c|ccc}
        & A1 & A2 & A3 & A4 & A5 & A6 & A7 & A8 \\ \hline
    A1 & $0$  & $\sqrt{45}$ & $\sqrt{63}$ & $\sqrt{57}$ & $\sqrt{41}$ & $\sqrt{28}$ & $\sqrt{95}$ & $\sqrt{6}$ \\
    A2 &    & $0$  & $\sqrt{55}$ & $\sqrt{49}$ & $\sqrt{35}$ & $\sqrt{11}$ & $\sqrt{5}$  & $\sqrt{25}$ \\
    A3 &    &    & $0$  & $\sqrt{11}$ & $\sqrt{23}$ & $\sqrt{54}$ & $\sqrt{47}$ & $\sqrt{65}$ \\ \hline
    A4 &    &    &    & $0$  & \textcolor{red}{$\sqrt{2}$} & $\sqrt{7}$  & $\sqrt{26}$ & $\sqrt{5}$  \\ \hline
    A5 &    &    &    &    & $0$  & $\sqrt{5}$  & $\sqrt{21}$ & $\sqrt{35}$ \\
    A6 &    &    &    &    &    & $0$  & $\sqrt{13}$ & $\sqrt{27}$ \\
    A7 &    &    &    &    &    &    & $0$  & $\sqrt{53}$ \\
    A8 &    &    &    &    &    &    &    & $0$ \\
    \end{tabular}
\end{table}

\begin{table}[h!]
    \centering
    \scriptsize
    \caption{Step 1 -- Clusters}
    \begin{tabular}{c|c|c}
    Level & \# Clusters & Clusters \\ \hline
    0     & 8           & $\{A1\}, \{A2\}, \{A3\}, \{A4\}, \{A5\}, \{A6\}, \{A7\}, \{A8\}$\\
    1     & 7           & $\{A1\}, \{A2\}, \{A3\}, \textcolor{red}{\{A4, A5\}}, \{A6\}, \{A7\}, \{A8\}$\\
    % 2     & 4           & \\
    % 3     & 3           & \\
    % 4     & 1           &   
    \end{tabular}
\end{table}

\begin{table}[h!]
    \centering
    \scriptsize
    \caption{Step 2}
    \begin{tabular}{c|cccccccc}
       & A1 & A2 & A3 & A4 & A5 & A6 & A7 & A8 \\ \hline
    A1 & $0$  & $\sqrt{45}$ & $\sqrt{63}$ & $\sqrt{57}$ & $\sqrt{41}$ & $\sqrt{28}$ & $\sqrt{95}$ & $\sqrt{6}$ \\
    A2 &    & $0$  & $\sqrt{55}$ & $\sqrt{49}$ & $\sqrt{35}$ & $\sqrt{11}$ & \textcolor{red}{$\sqrt{5}$}  & $\sqrt{25}$ \\
    A3 &    &    & $0$  & $\sqrt{11}$ & $\sqrt{23}$ & $\sqrt{54}$ & $\sqrt{47}$ & $\sqrt{65}$ \\ 
    A4 &    &    &    & $0$  & \textcolor{gray}{$\sqrt{2}$} & $\sqrt{7}$  & $\sqrt{26}$ & \textcolor{red}{$\sqrt{5}$}  \\
    A5 &    &    &    &    & $0$  & \textcolor{red}{$\sqrt{5}$}  & $\sqrt{21}$ & $\sqrt{35}$ \\
    A6 &    &    &    &    &    & $0$  & $\sqrt{13}$ & $\sqrt{27}$ \\
    A7 &    &    &    &    &    &    & $0$  & $\sqrt{53}$ \\
    A8 &    &    &    &    &    &    &    & $0$ \\
    \end{tabular}
\end{table}

\begin{table}[h!]
    \centering
    \scriptsize
    \caption{Step 2 -- Clusters}
    \begin{tabular}{c|c|c}
    Level & \# Clusters & Clusters \\ \hline
    0     & 8           & $\{A1\}, \{A2\}, \{A3\}, \{A4\}, \{A5\}, \{A6\}, \{A7\}, \{A8\}$\\
    1     & 7           & $\{A1\}, \{A2\}, \{A3\}, \{A4, A5\}, \{A6\}, \{A7\}, \{A8\}$\\
    2     & 4           & $\{A1\}, \textcolor{red}{\{A2, A7\}}, \{A3\}, \textcolor{red}{\{A4, A5, A6,A8\}}$\\
    % 3     & 3           & \\
    % 4     & 1           &   
    \end{tabular}
\end{table}

Step 3:

\begin{table}[h!]
    \centering
    \scriptsize
    \caption{Step 3}
    \begin{tabular}{c|cccccccc}
       & A1 & A2 & A3 & A4 & A5 & A6 & A7 & A8 \\ \hline
    A1 & $0$  & $\sqrt{45}$ & $\sqrt{63}$ & $\sqrt{57}$ & $\sqrt{41}$ & $\sqrt{28}$ & $\sqrt{95}$ & \textcolor{red}{$\sqrt{6}$} \\
    A2 &    & $0$  & $\sqrt{55}$ & $\sqrt{49}$ & $\sqrt{35}$ & $\sqrt{11}$ & \textcolor{gray}{$\sqrt{5}$}  & $\sqrt{25}$ \\
    A3 &    &    & $0$  & $\sqrt{11}$ & $\sqrt{23}$ & $\sqrt{54}$ & $\sqrt{47}$ & $\sqrt{65}$ \\ 
    A4 &    &    &    & $0$  & \textcolor{gray}{$\sqrt{2}$} & $\sqrt{7}$  & $\sqrt{26}$ & \textcolor{gray}{$\sqrt{5}$}  \\
    A5 &    &    &    &    & $0$  & \textcolor{gray}{$\sqrt{5}$}  & $\sqrt{21}$ & $\sqrt{35}$ \\
    A6 &    &    &    &    &    & $0$  & $\sqrt{13}$ & $\sqrt{27}$ \\
    A7 &    &    &    &    &    &    & $0$  & $\sqrt{53}$ \\
    A8 &    &    &    &    &    &    &    & $0$ \\
    \end{tabular}
\end{table}

\begin{table}[h!]
    \centering
    \scriptsize
    \caption{Step 3 -- Clusters}
    \begin{tabular}{c|c|c}
    Level & \# Clusters & Clusters \\ \hline
    0     & 8           & $\{A1\}, \{A2\}, \{A3\}, \{A4\}, \{A5\}, \{A6\}, \{A7\}, \{A8\}$\\
    1     & 7           & $\{A1\}, \{A2\}, \{A3\}, \{A4, A5\}, \{A6\}, \{A7\}, \{A8\}$\\
    2     & 4           & $\{A1\}, \{A2, A7\}, \{A3\}, \{A4, A5, A6,A8\}$\\
    3     & 3           & $\textcolor{red}{\{A1, A4, A5, A6,A8\}}, \{A2, A7\}, \{A3\}$\\
    % 4     & 1           &   
    \end{tabular}
\end{table}

Step 4:

\begin{table}[h!]
    \centering
    \scriptsize
    \caption{Step 4}
    \begin{tabular}{c|cccccccc}
       & A1 & A2 & A3 & A4 & A5 & A6 & A7 & A8 \\ \hline
    A1 & $0$  & $\sqrt{45}$ & $\sqrt{63}$ & $\sqrt{57}$ & $\sqrt{41}$ & $\sqrt{28}$ & $\sqrt{95}$ & \textcolor{gray}{$\sqrt{6}$} \\
    A2 &    & $0$  & $\sqrt{55}$ & $\sqrt{49}$ & $\sqrt{35}$ & \textcolor{red}{$\sqrt{11}$} & \textcolor{gray}{$\sqrt{5}$}  & $\sqrt{25}$ \\
    A3 &    &    & $0$  & \textcolor{red}{$\sqrt{11}$} & $\sqrt{23}$ & $\sqrt{54}$ & $\sqrt{47}$ & $\sqrt{65}$ \\ 
    A4 &    &    &    & $0$  & \textcolor{gray}{$\sqrt{2}$} & \textcolor{blue}{$\sqrt{7}$}  & $\sqrt{26}$ & \textcolor{gray}{$\sqrt{5}$}  \\
    A5 &    &    &    &    & $0$  & \textcolor{gray}{$\sqrt{5}$}  & $\sqrt{21}$ & $\sqrt{35}$ \\
    A6 &    &    &    &    &    & $0$  & $\sqrt{13}$ & $\sqrt{27}$ \\
    A7 &    &    &    &    &    &    & $0$  & $\sqrt{53}$ \\
    A8 &    &    &    &    &    &    &    & $0$ \\
    \end{tabular}
\end{table}

\begin{table}[h!]
    \centering
    \scriptsize
    \caption{Step 4 -- Clusters. Note that A4 and A6 are already in one cluster.}
    \begin{tabular}{c|c|c}
    Level & \# Clusters & Clusters \\ \hline
    0     & 8           & $\{A1\}, \{A2\}, \{A3\}, \{A4\}, \{A5\}, \{A6\}, \{A7\}, \{A8\}$\\
    1     & 7           & $\{A1\}, \{A2\}, \{A3\}, \{A4, A5\}, \{A6\}, \{A7\}, \{A8\}$\\
    2     & 4           & $\{A1\}, \{A2, A7\}, \{A3\}, \{A4, A5, A6,A8\}$\\
    3     & 3           & \textcolor{blue}{$\{A1, A4, A5, A6,A8\}, \{A2, A7\}, \{A3\}$}\\
    4     & 1           & $\{A1, A2, A3, A4, A5, A6, A7, A8\}$
    \end{tabular}
\end{table}


\begin{figure}[h!]
    \centering
    \scriptsize
    \includegraphics[width=0.45\textwidth]{dendrogram.png}
    \caption{Step 5 -- Dendrogram: Make sure you choose the correct order to start with. Use the sequence from the second last level;}
\end{figure}

\newpage
\section{Question 2 -- Ooutlier/Anomaly Detection}
\label{q2}

\subsection{}
What are the three types of anomaly? Give an example for each type.

\noindent
\textbf{Answer:}

\begin{itemize}
    \item \textbf{Global outlier (Point anomaly):} deviates significantly from the rest of the data set. 
    The simplest type of outliers.
    \item \textbf{Contextual outlier (Conditional outlier):} deviates significantly with respect to a specific context of the object.
        \begin{itemize}
            \item \textbf{Contextual attributes:} define the object’s context.
            \item \textbf{Behavioral attributes:} define the object’s characteristics, and are used to evaluate
            whether the object is an outlier in the context.\\
            \textbf{For example:}\\
            A temperature sensor measures 4$^{\circ}$C in May. It is a perfectly normal reading in Wellington, but it might be an outlier if the location is New York.
            The location and the date are \textbf{contextual attributes}, and the temperature is a \textbf{behavioral attribute}.
        
    \end{itemize}
    \item \textbf{Collective outliers:} the objects as a whole deviate significantly from the entire data set.
\end{itemize}

\begin{figure}[h!]
    \centering
    \includegraphics[width=0.7\textwidth]{outlier_examples.pdf}
    \caption{Types of outliers}
    \label{fig.outliers}
\end{figure}

Examples:
\begin{itemize}
    \item Fig.~\ref{fig.outliers} left: Global outliers
    \item Fig.~\ref{fig.outliers} middle: Contextual outliers -- Given the dataset has two clusters, each one has a moon shape.
    \item Fig.~\ref{fig.outliers} rright: Blue points are collective outliers because the density of those points is much higher than the rest.
\end{itemize}

\pagebreak
\subsection{}
You are given the following list of 2D data points:
\[[1; 1]; [1; 2]; [2; 2]; [2; 1]; [3; 3]; [2; 5]; [2; 3]\]
If you had to select one point to be anomalous, how to use Manhattan distance to 
determine the outlier. Explain the anomaly detection technique.

\noindent
\textbf{Answer:}

There are multiple ways to solve this problem. Let's use \textbf{distance-based outlier detection}.

\begin{table}[h!]
    \scriptsize
    \centering
    \begin{tabular}{c|ccccccc}
                 & \textbf{1,1} & \textbf{1,2} & \textbf{2,2} & \textbf{2,1} & \textbf{3,3} & \textbf{2,5} & \textbf{2,3} \\ \hline
    \textbf{1,1} & 0            & 1            & 2            & 1            & 4            & 5            & 3            \\
    \textbf{1,2} & 1            & 0            & 1            & 2            & 3            & 4            & 2            \\
    \textbf{2,2} & 2            & 1            & 0            & 1            & 2            & 3            & 1            \\
    \textbf{2,1} & 1            & 2            & 1            & 0            & 3            & 4            & 2            \\
    \textbf{3,3} & 4            & 3            & 2            & 3            & 0            & 3            & 1            \\
    \textbf{2,5} & 5            & 4            & 3            & 4            & 3            & 0            & 2            \\
    \textbf{2,3} & 3            & 2            & 1            & 2            & 1            & 2            & 0           
    \end{tabular}
    \caption{Manhattan Distance Matrix}
\end{table}

\[
    \frac{|\{o' | \text{dist}(o, o') \le r\}|}{|D|} \le \pi
\]
where $|D|$ is the {\em cardinality} of the set $D$.\\
\hspace*{1em}

We need to define the hyperparameters: $r$ and $\pi$.

The number of data points, $|D|$, is 7.\\
Let $r=2$,

\begin{table}[h!]
    \scriptsize
    \centering
    \begin{tabular}{c|c|c}
                    &  \textbf{\# of objects within $r$} & \textbf{Divide by $|D|$}\\ \hline
    \textbf{1,1} &  3 & 0.43 \\
    \textbf{1,2} &  4 & 0.57 \\
    \textbf{2,2} &  5 & 0.71 \\
    \textbf{2,1} &  4 & 0.57 \\
    \textbf{3,3} &  2 & 0.29 \\
    \rowcolor[HTML]{34CDF9}\textbf{2,5} &  1 & 0.14 \\
    \textbf{2,3} &  5 & 0.71   
    \end{tabular}
    \caption{The \# of neighbours within $r$}
\end{table}

If we select only one point as an outlier, we can set $\pi$ to any value between $0.14$ and $0.29$, e.g. $0.15$.\\
Therefore, we identify $[2;5]$ is an outlier.

\pagebreak
\subsection{}
Consider a set of points (0,0), (1,0), (0,1), (3,0). 
Calculate the {\em Local Outlier Factor} (LOF) score for the points using Manhattan distance and $k$ is 2.

\noindent
\textbf{Answer:}

\begin{figure}[h!]
    \centering
    \includegraphics[width=0.3\textwidth]{outlier_lof.pdf}
\end{figure}

\begin{table}[h!]
    \scriptsize
    \centering
    \begin{tabular}{c|cccc}
                 & \textbf{a (0,0)} & \textbf{b (1,0)} & \textbf{c (0,1)} & \textbf{d (3,0)} \\ \hline
    \textbf{a (0,0)} & 0            & 1            & 1            & 3            \\
    \textbf{b (1,0)} & 1            & 0            & 2            & 2            \\
    \textbf{c (0,1)} & 1            & 2            & 0            & 4            \\
    \textbf{d (3,0)} & 3            & 2            & 4            & 0           
    \end{tabular}
    \caption{Manhattan distance matrix}
\end{table}

\begin{table}[h!]
    \scriptsize
    \centering
    \begin{tabular}{c|cc}
                 & $\text{dist}_2(o)$ & $N_2(o)$ \\ \hline
    \textbf{a (0,0)} & 1 & 2 \\
    \textbf{b (1,0)} & 2 & \textcolor{red}{3} \\
    \textbf{c (0,1)} & 2 & 2 \\
    \textbf{d (3,0)} & 3 & 2  
    \end{tabular}
    \caption{Distance between data point $o$ and its k-th nearest neighbour ($k=2$)}
\end{table}
We denote the set of $k$-nearest neighbours as $N_k(o)$:
\[
    N_k(o) = \{o' | o' \in D, \text{dist}(o, o') \le \text{dist}_k(o)\}
\]
\textbf{Note:} $|N_k(o)|$ may contain more than $k$ objects, because objects may have same distance.

Let $o'$ be a neighbour of $o$, to avoid $\text{dist}(o, o')$ is too small, we compute the reachability distance from $o$ to $o'$:
\[
    \text{reachdist}_k(o' \leftarrow o) = \max\{\text{dist}_k(o), \text{dist}(o, o')\}
\]

Note that reachability distance is not symmetric, thus 
\[
    \text{reachdist}_k(o \leftarrow o') \neq \text{reachdist}_k(o' \leftarrow o)
\]

\begin{figure}
    \centering
    \includegraphics[width=0.33\textwidth]{reachdist_ab.pdf}
    \caption{$\text{reachdist}_2(b \leftarrow a) \neq \text{reachdist}_2(a \leftarrow b)$}
\end{figure}

\begin{equation*}
    \begin{array}{rlll}
        \rowcolor[HTML]{34CDF9} \text{reachdist}_2(b \leftarrow a) & = \max\{\text{dist}_2(a), \text{dist}(a, b)\} & = \max\{1, 1\} & = 1\\
        \text{reachdist}_2(c \leftarrow a) & = \max\{\text{dist}_2(a), \text{dist}(a, c)\} & = \max\{1, 1\} & = 1\\
        \text{reachdist}_2(d \leftarrow a) & = \max\{\text{dist}_2(a), \text{dist}(a, d)\} & = \max\{1, 3\} & = 3\\
        \rowcolor[HTML]{34CDF9} \text{reachdist}_2(a \leftarrow b) & = \max\{\text{dist}_2(b), \text{dist}(b, a)\} & = \max\{2, 1\} & = 2\\
        \text{reachdist}_2(c \leftarrow b) & = \max\{\text{dist}_2(b), \text{dist}(b, c)\} & = \max\{2, 2\} & = 2\\
        \text{reachdist}_2(d \leftarrow b) & = \max\{\text{dist}_2(b), \text{dist}(b, d)\} & = \max\{2, 2\} & = 2\\
        \text{reachdist}_2(a \leftarrow c) & = \max\{\text{dist}_2(c), \text{dist}(c, a)\} & = \max\{2, 1\} & = 2\\
        \text{reachdist}_2(b \leftarrow c) & = \max\{\text{dist}_2(c), \text{dist}(c, b)\} & = \max\{2, 2\} & = 2\\
        \text{reachdist}_2(d \leftarrow c) & = \max\{\text{dist}_2(c), \text{dist}(c, d)\} & = \max\{2, 4\} & = 4\\
        \text{reachdist}_2(a \leftarrow d) & = \max\{\text{dist}_2(d), \text{dist}(d, a)\} & = \max\{3, 3\} & = 3\\
        \text{reachdist}_2(b \leftarrow d) & = \max\{\text{dist}_2(d), \text{dist}(d, b)\} & = \max\{3, 2\} & = 3\\
        \text{reachdist}_2(c \leftarrow d) & = \max\{\text{dist}_2(d), \text{dist}(d, c)\} & = \max\{3, 4\} & = 4\\
    \end{array}
\end{equation*}

Note that,
\[
    \text{reachdist}_k(o \leftarrow o') \neq \text{reachdist}_k(o' \leftarrow o)
\]

The {\em Local Reachability Density} (LRD) of an object $o$ is defined as
\[
    \text{lrd}_k(o) = \frac{|N_k(o)|}{\sum_{o' \in N_k(o)} \text{reachdist}_k(o \leftarrow o')}
    \]
    
\begin{equation*}
    \begin{array}{rlll}
        \text{lrd}_2(a) & = |N_2(a)| / (\text{reachdist}_2(a \leftarrow b) + \text{reachdist}_2(a \leftarrow c))                                      & = 2 / (2 + 2) & = 0.5\\
        \text{lrd}_2(b) & = |N_2(b)| / (\text{reachdist}_2(b \leftarrow a) + \text{reachdist}_2(b \leftarrow c) + \text{reachdist}_2(b \leftarrow d)) & = 3 / (1+2+3) & = 0.5\\
        \text{lrd}_2(c) & = |N_2(c)| / (\text{reachdist}_2(c \leftarrow a) + \text{reachdist}_2(c \leftarrow b))                                      & = 2 / (1 + 2) & = 0.667\\
        \text{lrd}_2(d) & = |N_2(d)| / (\text{reachdist}_2(d \leftarrow a) + \text{reachdist}_2(d \leftarrow b))                                      & = 2 / (3 + 2) & = 0.4\\
    \end{array}
\end{equation*}

The {\em Local Outlier Factor} (LOF) of an object $o$ is
\[
    \text{LOF}_k(o) = \frac{\sum_{o' \in N_k(o)} \frac{\text{lrd}_k(o')}{\text{lrd}_k(o)}}{|N_k(o)|} 
    = \frac{\sum_{o' \in N_k(o)} \text{lrd}_k(o')}{|N_k(o)| \cdot \text{lrd}_k(o)}
\]
    
\begin{equation*}
    \begin{array}{rlll}
        \text{LOF}_2(a) & = (\text{lrd}_2(b) + \text{lrd}_2(c)) / (|N_2(a)| \cdot \text{lrd}_2(a)) & = (0.5+0.667) / (2\times 0.5)& = 1.167\\
        \text{LOF}_2(b) & = (\text{lrd}_2(a) + \text{lrd}_2(c) + \text{lrd}_2(d)) / (|N_2(b)| \cdot \text{lrd}_2(b))  & = (0.5+0.667+0.4) / (3\times 0.5)& \approx 1.045\\
        \text{LOF}_2(c) & = (\text{lrd}_2(a) + \text{lrd}_2(b)) / (|N_2(c)| \cdot \text{lrd}_2(c))  & = (0.5+0.5) / (2\times 0.667)& \approx 0.750\\
        \text{LOF}_2(d) & = (\text{lrd}_2(a) + \text{lrd}_2(b)) / (|N_2(d)| \cdot \text{lrd}_2(d))  & = (0.5+0.5) / (2\times 0.4)& = 1.25\\
    \end{array}
\end{equation*}

LOF is a relative value. We should not compare a LOF from one dataset with other reading from different datasets.

The question does not ask for outlier. There is only 4 data points. If we remove 1 point, we will remove $25\%$ of the data. I do not determine the outlier here.
\end{document}
