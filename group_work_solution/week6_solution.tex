\documentclass[10pt]{article}

\usepackage[english]{babel}
\usepackage[linguistics]{forest}
\usepackage[utf8]{inputenc}
\usepackage{algorithmic}
\usepackage{amsfonts}
\usepackage{amsmath}
\usepackage{amssymb}
\usepackage{array}
\usepackage{bookmark}
\usepackage{caption}
\usepackage{colortbl}
\usepackage{csquotes}
\usepackage{graphicx}
\usepackage{hyperref}
\usepackage{lipsum}
\usepackage{lmodern}
\usepackage{mathptmx}
\usepackage{mathtools}
\usepackage{multirow}
\usepackage{pgfplots}
\usepackage{svg}
\usepackage{xcolor}
\usepackage{multicol}

\usetikzlibrary{calc}

\DeclareMathOperator*{\argmax}{argmax}

\makeatletter
\def\@seccntformat#1{%
  \expandafter\ifx\csname c@#1\endcsname\c@section\else
  \csname the#1\endcsname\quad
  \fi}
\makeatother

\topmargin -.5in
\textheight 9in
\oddsidemargin -.25in
\evensidemargin -.25in
\textwidth 7in

\sloppy

\begin{document}

\title{COMPSCI 762 2022 S1 Week 6 Solution}
\author{Luke Chang}

\maketitle

\medskip

\section{Question 1}
\label{q1}

\begin{itemize}
    \item Apply binning with mean value to split the data into 3 bins based on ``Year'':
    \begin{table}[h]
        \small
        \centering
        \begin{tabular}{c|c|c|c}
        \textbf{Bin} & \textbf{Mean (Year, Price)} & \textbf{SD (Year, Price)} & \textbf{2SD (Year, Price)}\\ \hline
        1 & 1994, 11520 & 8.0, 6761.1 & 16.0, 13522.2 \\
        2 & 2008, 15100 & 1.2, 1546.0 &  2.4, 3092.0 \\
        3 & 2015, 20180 & 2.1, 7371.4 &  4.2, 14742.8 \\
        \end{tabular}
    \end{table}
    \item Absolute error:
    \begin{table}[h]
        \scriptsize
        \centering
        \begin{tabular}{c|llllllllllllllllllll}
          Year  &  9.0 &     4.0 &      3.0 &     7.0 &    10.0 &    1.0 &    1.0 &    0.0 &     0.0 &     2.0 &    2.0 &      1.0 &     0.0 &     2.0 &     3.0 \\
          \hline
          Price & 280.0 &  4920.0 &  10280.0 &  7120.0 &  1480.0 &  500.0 &  400.0 &  500.0 &  1900.0 &  2300.0 &  980.0 &  11980.0 &  1320.0 &  7120.0 &  4520.0 \\
        \end{tabular}
      \end{table}
    \item Based on 95\% confidence interval (2$\sigma$), there is no outlier in the data.
    \item The car with lowest price (\$4400) seems way cheaper than others. Why isn't it an outlier?\\
        \textbf{Note:} 
        \begin{itemize}
            \item There are multiple ways determine an outlier. We have to based on a statistical test, not an arbitrary guess. If the criteria we have defined is 95\% confidence interval, then none of the data point is outlier.
            \item The 1st bin contains older vehicles, and we observe it fluctuates more than other bins. 
        \end{itemize}
    \item If we model the data using least square method ($y=ax+b$), linear transformation will not alter the value of gradient ($a$). It only changes the bias ($b$). 
    \item $f(x) = 313.1 x - 612397.6$, thus $f(2022) = 20690.6$;
\end{itemize}

\section{Question 2}
\label{q2}

\begin{itemize}
    \item Available imputation methods: Median, mode.
    \item ``Engine size'' is discrete value. We shouldn't use mean.
    \item KNN imputation is not suitable in this case, unless we can prove ``Year'' is correlated to ``Engine size''.
    \item If the target of the regression model is ``Price'', we cannot use ``Price'' in imputation, because we will not have the target value at inference time.
    \item Median and mode both give us $2.0$.
    \item Spearman rank correlation in \texttt{pandas.DataFrame.corr} can be used.
    \item Mean price for all $15600$; Mean value exclude NA $17900$; Mean value for NA only $6400$. However, the sample size is too small to consider SD.
    \item Suggestion: Adding additional feature of ``Unknown engine size''.
    \item Ground truth: Engine size is randomly generated, but unknown engine size indicates it is a wrecked car.
    \item Blindly including extra features may not improve the performance of the model. 
\end{itemize}

\end{document}
